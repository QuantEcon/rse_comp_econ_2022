\documentclass[
    xcolor={svgnames},
    hyperref={colorlinks, citecolor=DeepPink4, linkcolor=DarkRed, urlcolor=DarkBlue}
    ]{beamer}  % for hardcopy add 'trans'

\mode<presentation>
{
  \usetheme{Singapore}
  % or ...
  \setbeamercovered{transparent}
  % or whatever (possibly just delete it)
}

\usefonttheme{professionalfonts}
%\usepackage[english]{babel}
% or whatever
%\usepackage[latin1]{inputenc}
% or whatever
%\usepackage{times}
%\usepackage[T1]{fontenc}
% Or whatever. Note that the encoding and the font should match. If T1
% does not look nice, try deleting the line with the fontenc.

%\usepackage{fontspec}
%\setmonofont{CMU Typewriter Text}
%\setmonofont{Consolas}

%%%%%%%%%%%%%%%%%%%%%% start my preamble %%%%%%%%%%%%%%%%%%%%%%

\addtobeamertemplate{navigation symbols}{}{%
    \usebeamerfont{footline}%
    \usebeamercolor[fg]{footline}%
    \hspace{1em}%
    \insertframenumber/\inserttotalframenumber
}


\usepackage{graphicx}
\usepackage{amsmath, amssymb, amsthm}
\usepackage{bbm}
\usepackage{mathrsfs}
\usepackage{xcolor}
\usepackage{fancyvrb}

% Quotes at start of chapters / sections
\usepackage{epigraph}  
%\renewcommand{\epigraphflush}{flushleft}
%\renewcommand{\sourceflush}{flushleft}
\renewcommand{\epigraphwidth}{6in}

%% Fonts

%\usepackage[T1]{fontenc}
\usepackage{mathpazo}
%\usepackage{fontspec}
%\defaultfontfeatures{Ligatures=TeX}
%\setsansfont[Scale=MatchLowercase]{DejaVu Sans}
%\setmonofont[Scale=MatchLowercase]{DejaVu Sans Mono}
%\setmathfont{Asana Math}
%\setmainfont{Optima}
%\setmathrm{Optima}
%\setboldmathrm[BoldFont={Optima ExtraBlack}]{Optima Bold}

% Some colors

\definecolor{aquamarine}{RGB}{69,139,116}
\definecolor{midnightblue}{RGB}{25,25,112}
\definecolor{darkslategrey}{RGB}{47,79,79}
\definecolor{darkorange4}{RGB}{139,90,0}
\definecolor{dogerblue}{RGB}{24,116,205}
\definecolor{blue2}{RGB}{0,0,238}
\definecolor{bg}{rgb}{0.95,0.95,0.95}
\definecolor{DarkOrange1}{RGB}{255,127,0}


\setlength{\parskip}{1.5ex plus0.5ex minus0.5ex}

%\renewcommand{\baselinestretch}{1.05}
%\setlength{\parskip}{1.5ex plus0.5ex minus0.5ex}
%\setlength{\parindent}{0pt}

% Typesetting code
\definecolor{bg}{rgb}{0.95,0.95,0.95}
\usepackage{minted}
\setminted{mathescape, frame=lines, framesep=3mm}
\usemintedstyle{friendly}
%\newminted{python}{}
%\newminted{c}{mathescape,frame=lines,framesep=4mm,bgcolor=bg}
%\newminted{java}{mathescape,frame=lines,framesep=4mm,bgcolor=bg}
%\newminted{julia}{mathescape,frame=lines,framesep=4mm,bgcolor=bg}
%\newminted{ipython}{mathescape,frame=lines,framesep=4mm,bgcolor=bg}


\newcommand{\Fact}{\textcolor{Brown}{\bf Fact. }}
\newcommand{\Facts}{\textcolor{Brown}{\bf Facts }}
\newcommand{\keya}{\textcolor{turquois4}{\bf Key Idea. }}
\newcommand{\Factnodot}{\textcolor{Brown}{\bf Fact }}
\newcommand{\Eg}{\textcolor{ForestGreen}{Example. }}
\newcommand{\Egs}{\textcolor{ForestGreen}{Examples. }}
\newcommand{\Ex}{{\bf Ex. }}



\renewcommand{\theFancyVerbLine}{\sffamily
    \textcolor[rgb]{0.5,0.5,1.0}{\scriptsize {\arabic{FancyVerbLine}}}}

\newcommand{\navy}[1]{\textcolor{Blue}{\bf #1}}
\newcommand{\brown}[1]{\textcolor{Brown}{\sf #1}}
\newcommand{\green}[1]{\textcolor{ForestGreen}{\sf #1}}
\newcommand{\blue}[1]{\textcolor{Blue}{\sf #1}}
\newcommand{\navymth}[1]{\textcolor{Blue}{#1}}
\newcommand{\emp}[1]{\textcolor{DarkOrange1}{\bf #1}}
\newcommand{\red}[1]{\textcolor{Red}{\bf #1}}

% Symbols, redefines, etc.

\newcommand{\code}[1]{\texttt{#1}}

\newcommand{\argmax}{\operatornamewithlimits{argmax}}
\newcommand{\argmin}{\operatornamewithlimits{argmin}}

\DeclareMathOperator{\cl}{cl}
\DeclareMathOperator{\interior}{int}
\DeclareMathOperator{\Prob}{Prob}
\DeclareMathOperator{\determinant}{det}
\DeclareMathOperator{\trace}{trace}
\DeclareMathOperator{\Span}{span}
\DeclareMathOperator{\rank}{rank}
\DeclareMathOperator{\cov}{cov}
\DeclareMathOperator{\corr}{corr}
\DeclareMathOperator{\var}{var}
\DeclareMathOperator{\mse}{mse}
\DeclareMathOperator{\se}{se}
\DeclareMathOperator{\row}{row}
\DeclareMathOperator{\col}{col}
\DeclareMathOperator{\range}{rng}
\DeclareMathOperator{\dimension}{dim}
\DeclareMathOperator{\bias}{bias}


% mics short cuts and symbols
\newcommand{\st}{\ensuremath{\ \mathrm{s.t.}\ }}
\newcommand{\setntn}[2]{ \{ #1 : #2 \} }
\newcommand{\cf}[1]{ \lstinline|#1| }
\newcommand{\fore}{\therefore \quad}
\newcommand{\tod}{\stackrel { d } {\to} }
\newcommand{\toprob}{\stackrel { p } {\to} }
\newcommand{\toms}{\stackrel { ms } {\to} }
\newcommand{\eqdist}{\stackrel {\textrm{ \scriptsize{d} }} {=} }
\newcommand{\iidsim}{\stackrel {\textrm{ {\sc iid }}} {\sim} }
\newcommand{\1}{\mathbbm 1}
\newcommand{\dee}{\,{\rm d}}
\newcommand{\given}{\, | \,}
\newcommand{\la}{\langle}
\newcommand{\ra}{\rangle}

\newcommand{\boldA}{\mathbf A}
\newcommand{\boldB}{\mathbf B}
\newcommand{\boldC}{\mathbf C}
\newcommand{\boldD}{\mathbf D}
\newcommand{\boldM}{\mathbf M}
\newcommand{\boldP}{\mathbf P}
\newcommand{\boldQ}{\mathbf Q}
\newcommand{\boldI}{\mathbf I}
\newcommand{\boldX}{\mathbf X}
\newcommand{\boldY}{\mathbf Y}
\newcommand{\boldZ}{\mathbf Z}

\newcommand{\bSigmaX}{ {\boldsymbol \Sigma_{\hboldbeta}} }
\newcommand{\hbSigmaX}{ \mathbf{\hat \Sigma_{\hboldbeta}} }

\newcommand{\RR}{\mathbbm R}
\newcommand{\NN}{\mathbbm N}
\newcommand{\PP}{\mathbbm P}
\newcommand{\EE}{\mathbbm E \,}
\newcommand{\XX}{\mathbbm X}
\newcommand{\ZZ}{\mathbbm Z}
\newcommand{\QQ}{\mathbbm Q}

\newcommand{\fF}{\mathcal F}
\newcommand{\dD}{\mathcal D}
\newcommand{\lL}{\mathcal L}
\newcommand{\gG}{\mathcal G}
\newcommand{\hH}{\mathcal H}
\newcommand{\nN}{\mathcal N}
\newcommand{\pP}{\mathcal P}




\title{Computational Modeling for Economists}

\subtitle{An Introduction using Python}

\author{A QuantEcon--RSE Collaboration}

\date{February 2020}


\begin{document}

\begin{frame}
  \titlepage
\end{frame}





\section{Schedule}



\begin{frame}
    \frametitle{Introduction}


    \textbf{Lecturers}

    \begin{itemize}
        \item \green{Matt McKay}
            \vspace{0.5em}
        \item \green{John Stachurski}
            \vspace{0.5em}
    \end{itemize}

            \vspace{0.5em}
            \vspace{0.5em}

    \textbf{Plan}

    \begin{itemize}
        \item Review introductory QuantEcon lectures 
    \end{itemize}



            \vspace{0.5em}
            \vspace{0.5em}

    \textbf{Thanks}


    \begin{itemize}
        \item Alfred P. Sloan Foundation, Research School of Economics
    \end{itemize}

\end{frame}




\begin{frame}
    \frametitle{Downloads / Installation }

    \green{Install Python + Scientific Libs}
    
    \begin{itemize}
        \item Install Anaconda from {\footnotesize \url{https://www.anaconda.com/downloads}}
        \vspace{1em}
            \begin{itemize}
                \item Select latest Python version (3.7)
                \item For your OS!
            \end{itemize}
        \vspace{1em}
        \item Not plain vanilla Python
    \end{itemize}


    \vspace{1em}

    \green{Remote options}

    \begin{itemize}
        \item \url{https://colab.research.google.com}
        \item etc.
    \end{itemize}


\end{frame}






\begin{frame}
    \frametitle{Prereqs / Aims / Outcomes }

    Assumptions:

    \begin{itemize}
        \item econ/computer/maths/stats literate
        \vspace{0.3em}
        \item no familiarity with Python
    \end{itemize}

    \vspace{0.3em}
    \vspace{0.3em}

    Aims:
    %
    \begin{itemize}
        \item Overview of scientific computing and Python
            \vspace{0.3em}
        \item Review some simple economic models
            \vspace{0.3em}
        \item Show how to solve such models with Python
            \vspace{0.3em}
        \item Prep for remainder of the course
    \end{itemize}

\end{frame}




\section{Overview}


\begin{frame}
    \frametitle{Background --- Language Types}
    
    \blue{Proprietary} 
    %
    \begin{itemize}
        \item Excel
        \item MATLAB
        \item STATA, etc.
    \end{itemize}
    

    \vspace{0.5em}
    \vspace{0.5em}
    \blue{Open Source} 
    
    \begin{itemize}
        \item Python
        \item Julia
        \item R
    \end{itemize}


    \begin{center}
        closed and stable vs open and fast moving
    \end{center}

\end{frame}





\begin{frame}
    \frametitle{Background --- Language Types}
    
    \blue{Low level } 
    
    \begin{itemize}
        \item C/C++
        \item Fortran
        \item Java
    \end{itemize}

    \vspace{1em}

    \blue{High level } 

    \begin{itemize}
        \item Python
        \item Ruby
        \item Javascript
    \end{itemize}

\end{frame}




\begin{frame}[fragile]

    Low level languages give us fine grained control 
    
    \Eg \brown{1 + 1} in assembly

    {\small
    \begin{minted}{as}
pushq   %rbp
movq    %rsp, %rbp
movl    $1, -12(%rbp)
movl    $1, -8(%rbp)
movl    -12(%rbp), %edx
movl    -8(%rbp), %eax
addl    %edx, %eax
movl    %eax, -4(%rbp)
movl    -4(%rbp), %eax
popq    %rbp
    \end{minted}
    }



\end{frame}


\begin{frame}
    
    \blue{High level languages} give us abstraction, automation, etc.

\end{frame}



\begin{frame}[fragile]

    \Eg Reading from a file in Python
    
    \begin{minted}{python}
    data_file = open("data.txt")
    for line in data_file:
        print(line.capitalize()) 
    data_file.close()
    \end{minted}

\end{frame}



\begin{frame}
    \frametitle{Trade-Offs}

    \begin{figure}
       \begin{center}
        \scalebox{.36}{\includegraphics{tradeoff.pdf}}
       \end{center}
    \end{figure}

\end{frame}



\begin{frame}[fragile]
    \frametitle{But what about scientific computing?}
    
    \navy{Requirements}

    \begin{itemize}
        \item \underline{\green{Productive}} --- easy to read, write, debug, explore
            \vspace{0.4em}
            \vspace{0.4em}
            \vspace{0.4em}
        \item \underline{\green{Fast}} computations
    \end{itemize}

\end{frame}




\begin{frame}
    \frametitle{Trade-Offs}

    \begin{figure}
       \begin{center}
        \scalebox{.36}{\includegraphics{tradeoff2.pdf}}
       \end{center}
    \end{figure}

\end{frame}


\begin{frame}
    \frametitle{Trade-Offs}

    \begin{figure}
       \begin{center}
        \scalebox{.36}{\includegraphics{tradeoff3.pdf}}
       \end{center}
    \end{figure}

\end{frame}


\begin{frame}
    \frametitle{Trade-Offs}

    \begin{figure}
       \begin{center}
        \scalebox{.36}{\includegraphics{tradeoff4.pdf}}
       \end{center}
    \end{figure}

\end{frame}




\section{Trends}

\begin{frame}
    \frametitle{Trend 1: Parallelization}

    CPU frequency (clock speed) growth is slowing

    \begin{figure}
       \begin{center}
        \scalebox{.22}{\includegraphics{processor_clock.png}}
       \end{center}
    \end{figure}

\end{frame}


\begin{frame}
    
    Chip makers have responded by developing multi-core processors

    \begin{figure}
       \begin{center}
        \scalebox{.2}{\includegraphics{dual_core.png}}
       \end{center}
    \end{figure}

    Source: Wikipedia


\end{frame}


\begin{frame}

    \navy{GPUs / ASICs} are also becoming increasingly important


    \begin{figure}
       \begin{center}
        \scalebox{.16}{\includegraphics{gpu.jpg}}
       \end{center}
    \end{figure}

    \vspace{0.5em}

    Applications: machine learning, deep learning, etc.
    

\end{frame}


\begin{frame}
    \frametitle{Trend 2: Distributed Computing}
    
    Advantages: 
    %
    \begin{itemize}
        \item run code on big machines we don't have to buy
        \vspace{0.5em}
        \item customized execution environments
        \vspace{0.5em}
        \item circumvent internal IT departments
    \end{itemize}

    \vspace{0.5em}

    Options:
    %
    \begin{itemize}
        \item University machines
            \vspace{0.5em}
        \item AWS 
            \vspace{0.5em}
        \item Google Colab, etc.
    \end{itemize}

\end{frame}




\section{Why Python?}


\begin{frame}
    \frametitle{Why Python?}
    
    \begin{itemize}
        \item Easy to learn, well designed
            \vspace{0.5em}
        \item Massive scientific ecosystem
            \vspace{0.5em}
        \item Open source
            \vspace{0.5em}
        \item Huge demand for tech-savvy Python programmers
    \end{itemize}

\end{frame}


\begin{frame}
    \frametitle{Scientific Computing}
    
    Python has strong tools in vectorization / JIT compilation /
    parallelization / visualization / etc.

    Examples:

    \begin{itemize}
        \item SciPy, NumPy, Matplotlib, pandas
            \vspace{0.5em}
        \item Numba (JIT compilation, multithreading)
            \vspace{0.5em}
        \item Tensorflow, PyTorch (machine learning, AI)
            \vspace{0.5em}
        \item JAX, NetworkX, etc., etc.
    \end{itemize}

\end{frame}

\begin{frame}
    
    Python is convenient because it covers so many bases

            \vspace{0.5em}
            \vspace{0.5em}
    \begin{itemize}
        \item web dev, databases, system admin, GUIs
    \end{itemize}

            \vspace{0.5em}
            \vspace{0.5em}

    Chris Wiggins, Chief Data Scientist at The New York Times:

            \vspace{0.5em}
    \begin{center}
        Python has gotten sufficiently weapons grade that we don't descend into R anymore. Sorry, R people. I used to be one of you but we no longer descend into R.
    \end{center}


\end{frame}


\begin{frame}
    

    As a result of these advantages:

    \begin{figure}
       \begin{center}
        \scalebox{.4}{\includegraphics{python_vs_rest.png}}
       \end{center}
    \end{figure}


\end{frame}





\begin{frame}
    \frametitle{Using Jupyter Notebooks}

    \begin{itemize}
        \item Dashboard and notebooks
        \item Modal editing
        \item Arithmetic
        \item Strings
        \item Variables (including unicode)
        \item Built in functions
        \item Getting help
        \item Repetition: Introduction to loops
        \item Documentation and rich text (markdown, LaTeX)
    \end{itemize}

\end{frame}








\end{document}


